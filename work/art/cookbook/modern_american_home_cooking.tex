%Format  from: https://tex.stackexchange.com/questions/366229/an-aesthetically-pleasing-recipe-book-template
%Source recipes are a mix but originate from https://docs.google.com/document/d/1WrRI1nHxHD0yFOHVZtMPuJKYHQdceUM5kYcMDxSMnrw/edit#heading=h.bx3ugzbriux1
\documentclass{article}
\usepackage{fancyhdr,wrapfig,xcolor,graphicx,xparse,fontspec}
\usepackage[%
    %a5paper,
    papersize={5.5in,8.5in},
    margin=0.75in,
    top=0.75in,
    bottom=0.75in,
    %twoside
    ]{geometry}

\newcounter{stepnum}

%% |=====8><-----| %%


\makeatletter

%% From Donald Arseneau. Add after the wrapping text. Whew!
\def\wrapfill{% Just glad it works.
    \par
  \ifx\parshape\WF@fudgeparshape
    \nobreak
    \ifnum\c@WF@wrappedlines>\@ne
      \advance\c@WF@wrappedlines\m@ne
      \vskip\c@WF@wrappedlines\baselineskip
      \global\c@WF@wrappedlines\z@
    \fi
    \allowbreak
    \WF@finale
  \fi
}


%% Used for the headnote and in \showit
%% If the text is small it is placed on one line;
%% otherwise it is put into a raggedright paragraph.
\long\def\testoneline#1{%
  \sbox\@tempboxa{#1}%
  \ifdim \wd\@tempboxa <0.75\linewidth
        \begingroup
            \itshape
            #1\par
        \endgroup
  \else
    \parbox{0.75\linewidth}{\raggedright\itshape#1}%
    \par
  \fi
}

\newif\if@mainmatter \@mainmattertrue

%% Borrowed from book.cls
\newcommand\frontmatter{%
    \cleardoublepage
  \@mainmatterfalse
  \pagenumbering{roman}}
\newcommand\mainmatter{%
    \cleardoublepage
  \@mainmattertrue
  \pagenumbering{arabic}}
\makeatother

%% Vary the colors at will

\definecolor{vegcolor}{rgb}{0,0.5,0.2}
\colorlet{gfcolor}{brown}
\definecolor{frzcolor}{rgb}{0,0.8,0.8}
\definecolor{dessertcolor}{rgb}{0.5,0.2,0.1}
\definecolor{makeaheadcolor}{rgb}{0.5,0.5,0.6}

%% Thanks to alephzero for the excellent start:
%% #1 [optional headnote]; #2 Title of recipe; #3 [Initial instructions]
\NewDocumentCommand{\recipe}{o m o}{%
    \setcounter{stepnum}{0}%
    \newpage
    \thispagestyle{fancy}
    \lhead{}%
    \chead{}%
    \rhead{}%
    \lfoot{}%
    \rfoot{}%
    \section{#2}%
    \IfNoValueF{#1}{\begin{center}\testoneline{#1}\end{center}}
    \IfNoValueF{#3}{\noindent\emph{#3}\par\medskip}
}
\newcommand{\serves}[2][Serves]{%
    \chead{#1 #2}}
\newcommand{\dishtype}[1]{%
    \rhead{#1}%
}
\newcommand{\dishother}[1]{%
    \lhead{#1}%
}
\newcommand{\vegetarian}{%
    {\large\color{vegcolor}\textbf{V}}%
}
\newcommand{\glutenfree}{%
    {\large\color{gfcolor}\textbf{GF}}%
}
\newcommand{\freeze}{%
    {\large\color{frzcolor}\textbf{F}}%
}
\newcommand{\dessert}{%
    {\large\color{dessertcolor}\textbf{D}}%
}
\newcommand{\makeahead}{%
    {\large\color{makeaheadcolor}\textbf{M}}%
}
%% Optional arguments for alternate names for these:
\newcommand{\preptime}[2][Prep time]{%
    \lfoot{#1: #2}%
}
\newcommand{\cooktime}[2][Cook time]{%
    \rfoot{#1: #2}%
}
\newcommand{\temp}[1]{%
    #1°C}
%% Optional argument is the width of the graphic, default = 1in
\newcommand{\showpic}[3][1in]{%
    \begin{center}
        \bigskip
            \includegraphics[width=#1]{#2}%
            \par
            \medskip
            \testoneline{#3}%
            \par
    \end{center}%
}

\def\ucit#1{\uppercase{#1}}
\begingroup
    \lccode`~=`\^^M
    \lowercase{%
\endgroup%% Ingredient first, then measure; empty measure and/or unit = " . "
    %% *=column break; amount<space>ingredient
    \NewDocumentCommand{\ing}{u{ } u{ } u{~}}{% %% basically the same as: \def\ing#1 #2~{% requires xparse
        \noindent
        \if#1#2% Is a heading, a non-ingredient, in the ingredients block
            \emph{#3}~ % A heading
        \else % Amounts containing spaces <1 teaspoon> have to use '~' <1~teaspoon>
            \textbf{\ucit#3, }#1\if.#2\else\ #2\fi~ %
        \fi
    }%
}%

\NewDocumentEnvironment{step}{}{%
    \parindent0pt
    \leftskip0pt
    \begin{minipage}{\textwidth}
        \begin{wrapfigure}{r}{0pt}
            \kern-0.5em
            \vrule width 1pt\enskip
            \begin{minipage}{0.5\textwidth}
                \leftskip=1.5em
                \parindent=-1.5em
                \parskip=0.25em
                \obeylines
                    \everypar={\ing}
}{%
        \wrapfill
    \end{minipage}
    \medskip
}

\NewDocumentCommand{\method}{}{%
            \end{minipage}
        \end{wrapfigure}
        \rightskip0pt plus 2em
        \parskip0.25em
        \everypar={\llap{\stepcounter{stepnum}\hbox to 1.5em{\thestepnum.\hfill}}}
}

\setmainfont{STIX Two Text}

\pagestyle{plain}
\setlength{\intextsep}{0pt}

\begin{document}

\frontmatter
\tableofcontents

\mainmatter

\recipe[Sans wood fired grill, broiling tomatoes, onions, and peppers converts starchy carbohydrates into sweeter sugars.  Additionally the infrared heat wonderfully chars imparting a mild smokey flavor to the sauce]{Roja Salsa v15}[?????]
%\serves{6-8}
%\preptime{1 hour}
%\cooktime{1 hour}
\dishtype{\dessert}
\dishother{\glutenfree}

Cook time: 45 minutes \\
Prep time: 45 minutes \\
Special Equipment:  Blender (food processor works okay) \\\\

\begin{step}
. . Grilled/Broiled Vegetables:
2 lb ~8 red plum/Roma tomatoes (or two rinsed 22 oz cans in off season)
~4 serranos (Spice to taste!)
2 white onions
4 garlic cloves in husk
2 tablespoon of olive oil for broil 
2 tablespoons ~2 limes



\method
Cut peppers, tomatoes, and onion in half.
On a baking tray arrange tomatoes and onion in the middle and the peppers and garlic on the edges.
\\
Drizzle liberally with olive oil.
Broil on high using the top rack for about 10 minutes until softened and lightly charred.
Juice the lime and reserve.

Remove the garlic and peppers to cool. 
Add onion, tomatoes and half! of the peppers to the blender.
De-husk the garlic and add it as well.
Process until diced.
\end{step}

\begin{step}
. . Blending ingrediants:
2 tablespoons ~2 limes
food processing oil (1 tablespoon)
Cilantro (stems are fine)
1 teaspoon of salt (salt to taste!)
1.5 teaspoon of pepper
1.5 cup of chicken broth or 1 teaspoon of inosinate and 1.5 cups of water if vegetarian 
\method
Add in all below and process until smooth.
If the salsa is not spicy enough add in half a pepper at a time, processing evenly for a few seconds and tasting, until the heat is to the desired level.

Let the salsa rest overnight in the fridge for the flavors to fully diffuse (45 minutes in the freezer cools it as well [set a timer])
\end{step}


\end{document}


% Source code
%\documentclass{article}
%\usepackage{fancyhdr,wrapfig,xcolor,graphicx,xparse,fontspec}
%\usepackage[%
%    %a5paper,
%    papersize={5.5in,8.5in},
%    margin=0.75in,
%    top=0.75in,
%    bottom=0.75in,
%    %twoside
%    ]{geometry}
%
%\newcounter{stepnum}
%
%%% |=====8><-----| %%
%
%
%\makeatletter
%
%%% From Donald Arseneau. Add after the wrapping text. Whew!
%\def\wrapfill{% Just glad it works.
%    \par
%  \ifx\parshape\WF@fudgeparshape
%    \nobreak
%    \ifnum\c@WF@wrappedlines>\@ne
%      \advance\c@WF@wrappedlines\m@ne
%      \vskip\c@WF@wrappedlines\baselineskip
%      \global\c@WF@wrappedlines\z@
%    \fi
%    \allowbreak
%    \WF@finale
%  \fi
%}
%
%
%%% Used for the headnote and in \showit
%%% If the text is small it is placed on one line;
%%% otherwise it is put into a raggedright paragraph.
%\long\def\testoneline#1{%
%  \sbox\@tempboxa{#1}%
%  \ifdim \wd\@tempboxa <0.75\linewidth
%        \begingroup
%            \itshape
%            #1\par
%        \endgroup
%  \else
%    \parbox{0.75\linewidth}{\raggedright\itshape#1}%
%    \par
%  \fi
%}
%
%\newif\if@mainmatter \@mainmattertrue
%
%%% Borrowed from book.cls
%\newcommand\frontmatter{%
%    \cleardoublepage
%  \@mainmatterfalse
%  \pagenumbering{roman}}
%\newcommand\mainmatter{%
%    \cleardoublepage
%  \@mainmattertrue
%  \pagenumbering{arabic}}
%\makeatother
%
%%% Vary the colors at will
%
%\definecolor{vegcolor}{rgb}{0,0.5,0.2}
%\colorlet{gfcolor}{brown}
%\definecolor{frzcolor}{rgb}{0,0.8,0.8}
%\definecolor{dessertcolor}{rgb}{0.5,0.2,0.1}
%\definecolor{makeaheadcolor}{rgb}{0.5,0.5,0.6}
%
%%% Thanks to alephzero for the excellent start:
%%% #1 [optional headnote]; #2 Title of recipe; #3 [Initial instructions]
%\NewDocumentCommand{\recipe}{o m o}{%
%    \setcounter{stepnum}{0}%
%    \newpage
%    \thispagestyle{fancy}
%    \lhead{}%
%    \chead{}%
%    \rhead{}%
%    \lfoot{}%
%    \rfoot{}%
%    \section{#2}%
%    \IfNoValueF{#1}{\begin{center}\testoneline{#1}\end{center}}
%    \IfNoValueF{#3}{\noindent\emph{#3}\par\medskip}
%}
%\newcommand{\serves}[2][Serves]{%
%    \chead{#1 #2}}
%\newcommand{\dishtype}[1]{%
%    \rhead{#1}%
%}
%\newcommand{\dishother}[1]{%
%    \lhead{#1}%
%}
%\newcommand{\vegetarian}{%
%    {\large\color{vegcolor}\textbf{V}}%
%}
%\newcommand{\glutenfree}{%
%    {\large\color{gfcolor}\textbf{GF}}%
%}
%\newcommand{\freeze}{%
%    {\large\color{frzcolor}\textbf{F}}%
%}
%\newcommand{\dessert}{%
%    {\large\color{dessertcolor}\textbf{D}}%
%}
%\newcommand{\makeahead}{%
%    {\large\color{makeaheadcolor}\textbf{M}}%
%}
%%% Optional arguments for alternate names for these:
%\newcommand{\preptime}[2][Prep time]{%
%    \lfoot{#1: #2}%
%}
%\newcommand{\cooktime}[2][Cook time]{%
%    \rfoot{#1: #2}%
%}
%\newcommand{\temp}[1]{%
%    #1°C}
%%% Optional argument is the width of the graphic, default = 1in
%\newcommand{\showpic}[3][1in]{%
%    \begin{center}
%        \bigskip
%            \includegraphics[width=#1]{#2}%
%            \par
%            \medskip
%            \testoneline{#3}%
%            \par
%    \end{center}%
%}
%
%\def\ucit#1{\uppercase{#1}}
%\begingroup
%    \lccode`~=`\^^M
%    \lowercase{%
%\endgroup%% Ingredient first, then measure; empty measure and/or unit = " . "
%    %% *=column break; amount<space>ingredient
%    \NewDocumentCommand{\ing}{u{ } u{ } u{~}}{% %% basically the same as: \def\ing#1 #2~{% requires xparse
%        \noindent
%        \if#1#2% Is a heading, a non-ingredient, in the ingredients block
%            \emph{#3}~ % A heading
%        \else % Amounts containing spaces <1 teaspoon> have to use '~' <1~teaspoon>
%            \textbf{\ucit#3, }#1\if.#2\else\ #2\fi~ %
%        \fi
%    }%
%}%
%
%\NewDocumentEnvironment{step}{}{%
%    \parindent0pt
%    \leftskip0pt
%    \begin{minipage}{\textwidth}
%        \begin{wrapfigure}{r}{0pt}
%            \kern-0.5em
%            \vrule width 1pt\enskip
%            \begin{minipage}{0.5\textwidth}
%                \leftskip=1.5em
%                \parindent=-1.5em
%                \parskip=0.25em
%                \obeylines
%                    \everypar={\ing}
%}{%
%        \wrapfill
%    \end{minipage}
%    \medskip
%}
%
%\NewDocumentCommand{\method}{}{%
%            \end{minipage}
%        \end{wrapfigure}
%        \rightskip0pt plus 2em
%        \parskip0.25em
%        \everypar={\llap{\stepcounter{stepnum}\hbox to 1.5em{\thestepnum.\hfill}}}
%}
%
%\setmainfont{STIX Two Text}
%
%\pagestyle{plain}
%\setlength{\intextsep}{0pt}
%
%\begin{document}
%
%\frontmatter
%\tableofcontents
%
%\mainmatter
%
%\recipe[Some would say this is better than pie. It is certainly easier. And delicious. The original recipe came from Dorie Greenspan; this version also includes almond flour, suggested by King Arthur Baking.]{French Apple Cake}[Center a rack in the oven and preheat the oven to 350°F. Generously butter an 8-inch springform pan and put it on a baking sheet lined with a silicone baking mat or parchment paper.]
%\serves{6-8}
%\preptime{1 hour}
%\cooktime{1 hour}
%\dishtype{\dessert}
%\dishother{\glutenfree}
%
%\begin{step}
%. . Batter, the dry:
%1 cup AP (or GF) flour
%½ cup almond flour
%1 teaspoon baking powder
%½ teaspoon cinnamon
%¼ teaspoon nutmeg
%¼ teaspoon salt
%\method
%Whisk the flour, baking powder, spices, and salt together in small bowl.
%\end{step}
%
%\begin{step}
%4 large apples (if you can, choose 4 different kinds)
%\method
%Peel the apples, cut them in half and remove the cores. Cut the apples into 1- to 2-inch chunks.
%\end{step}
%
%\begin{step}
%. . Batter, the wet:
%2 large eggs
%¾ cup maple or brown sugar
%3 tablespoons dark rum
%½ teaspoon pure vanilla extract
%2--3 drops lemon extract
%8 tablespoons unsalted butter, melted and cooled
%\method
%In a medium bowl, beat the eggs with a whisk until they’re foamy. Pour in the sugar and whisk for a minute or so to blend. Whisk in the rum, vanilla, and lemon oil. Whisk in  the flour and when it is incorporated, add the melted butter, mixing gently so that you have a smooth, rather thick batter.
%
%Use a rubber spatula to fold-in the apples--it might look as if there isn't enough batter, but there is. Put the batter into the prepared pan, smoothing the top as much as possible. Bake for 55--65 minutes, or until a toothpick inserted  in the middle comes out clean.
%
%Let cool 30 minutes. Before removing the side of the springform pan, run a knife around the edge of the cake to make sure no apples stuck to the pan.
%\end{step}
%
%
%\end{document}
