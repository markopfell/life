%SOURCE
% Stylish Curriculum Vitae
% LaTeX Template
% Version 1.0 (18/7/12)
% This template has been downloaded from:  http://www.LaTeXTemplates.com
% Original author:  Stefano (http://stefano.italians.nl/)
% IMPORTANT: THIS TEMPLATE NEEDS TO BE COMPILED WITH XeLaTeX
% License:  CC BY-NC-SA 3.0 (http://creativecommons.org/licenses/by-nc-sa/3.0/)

\documentclass[a4paper, oneside, final]{scrartcl}

\usepackage{scrpage2} % Provides headers and footers configuration
\usepackage{titlesec} % Allows creating custom \section's
\usepackage{marvosym} % Allows the use of symbols
\usepackage{tabularx,colortbl} % Advanced table configurations
\usepackage{fontspec} % Allows font customization
\usepackage{hyperref} % URL support
\usepackage{silence} %Kill verbose terminal output

\defaultfontfeatures{Mapping=tex-text}
\setmainfont{Times}

\titleformat{\section}{\large\scshape\raggedright}{}{0em}{}[\titlerule]

\pagestyle{scrheadings} % Print the headers and footers on all pages

\addtolength{\voffset}{-0.5in} % Adjust the vertical offset - less whitespace at the top of the page
\addtolength{\textheight}{3cm} % Adjust the text height - less whitespace at the bottom of the page

\newcommand{\gray}{\rowcolor[gray]{.90}} % Custom highlighting for the work and education sections


%FOOTER
\renewcommand{\headfont}{\normalfont\rmfamily\scshape} % Letter spacing and font size
\cofoot{
\addfontfeature{LetterSpace=20.0}\fontsize{12.5}{17}\selectfont 
+1-530-848-8212\\
\href{mailto:markopfell@gmail.com}{markopfell@gmail.com}\\
\href{https://github.com/markopfell}{github.com/markopfell}\\
\href{https://www.linkedin.com/in/markopfell/}{linkedin.com/markopfell}
}

\begin{document}
\begin{center} % Center everything in the document

\pdfpageheight 11in 
\pdfpagewidth 8.5in


%HEADER
{\addfontfeature{LetterSpace=20.0}\fontsize{36}{36}\selectfont\scshape Mark Opfell} 
\vspace{1.16 cm} % Extra whitespace after the large name at the top


%SKILLS & QUALIFICATIONS
\section{Exposure, Skills, and Certifications}
\begin{tabular}{ @{} >{\bfseries}l @{\hspace{6ex}} l }
RF Standards & DVB-S2, CCSDS, LTE, FCC, ITU\\
RF Tools & VNA, SDR, GNU Radio, VSA\\%, Antenna Hats\\
%Groundstation Network & KSAT Lite\\
General Software Tools & Python, Git*, Linux, Bash\\%, Excel (Wizard)\\
Python Libraries & NumPy, Matplotlib, Scapy, Skimage\\%, SciPy\\
Networking & Wireshark, Cisco Networking Technician\\ %Blacksky, Leostella public knowledge 
Cloud & AWS EC2 \& S3\\ %Blacksky, Leostella public knowledge 
Volcano Ascents & Mount Rainier, Mount Baker, Mount Adams \\%Stawamus Chief (Squamish Buttress face), 5 Gallon Buckets \\
%Significant Descents & Camp Muir (Winter), Skykomish (Class III+) \\ 
\\
\end{tabular}

%WORK EXPERIENCE ... actually life experience at this point
%Years of experience 2025-2012 = "13 years"
%Running total ~= 12 years = 11 years + 6 months as of 6/1/2025
\section{Experience}

\begin{tabularx}{0.97\linewidth}{>{\raggedleft\scshape}p{2cm}X}
\gray Job Title & \textbf{RF Communications System Engineer}\\
\gray Employer & \textbf{Amazon: Kuiper} \hfill Redmond, WA\\
\gray Period & \textbf{July 2024 -- Present}\\
%California equivalent = $300,000
%Washington total yearly compensation: $280,00
%Year 1 sign on payment = $99,600 broken out and paid monthly
%Year 2 sign on payment = $74,000 broken out and paid monthly
%1,115 units of AMZ = $188/share on the New York Stock exchange, vesting: Year 1 = 5%, Year 2 = 15%, Year 2.5 = 20%, Year 3 = 20%, Year 3.5 = 20%, Year 4 = 20%.
%base $170,000 
%company_size=1,600,000
&
\vspace{-0.15 cm}
%Project Kuiper: Developing Ka-band payload phased array system test infrastructure analyzing complex wireless links for mass production.
Developing Ka-band payload phased array system test infrastructure. 
\\
\\
\end{tabularx}

\begin{tabularx}{0.97\linewidth}{>{\raggedleft\scshape}p{2cm}X}
\gray Job Title & \textbf{Lead Communication Systems Engineer}\\
\gray Employer & \textbf{Albedo} \hfill Remote \& Some Travel\\
\gray Period & \textbf{October 2021 -- March 2024}\\
%total yearly compensation: $198,682.21 + 90,000 stock valued at $3/share, paid $0.01/share w, company_size=20
%base $175,000   
&
\vspace{-0.15 cm}
Created, evaluated, and built space-to-ground digital communications payload (gigabit class) for high data rate visible and thermal sensors. Developed the mission data chain from modulated waveform to frames, packets, and connections. Analyzed and tested with: engineering model radios, GNU radio, technical deep dives into CCSDS and DVB-S2 standards, and Python code for the processing pipeline.
\newline
\newline
Procured, set up, coded, documented, and maintained FlatSat (hardware-in-the-loop) S-band telemetry \& command megabit speed links with a COTS SDR, spectrum analyzer, couplers, attenuators, and coax.  Defined and tested first contact ConOps. Architected and built out facility RF testing flow, and lab-to-cloud remote VPN network. 
\newline
\newline
Lead NGSO imaging satellite constellation ITU, and FCC 312 Schedule S regulatory filings.  Ran RF analysis support efforts with Python scripts, and ITU Spacecap.  Collaborated with orbital dynamics expert, and mechanical engineers to decompose legal wording into requirements for satellite architecture and material choices ensuring proper post mission disposal. 
\newline
\newline
%\newline
%\newline
%Continuously building consensus with the founders (CTO, CEO, and CPO) on which space and ground communication business partnerships to pursue. Collaboration on product (satellite imagery) level vision, creation of a 10 year communications technology roadmap, and month-to-month schedule to break down a work plan to get launch while burning down realistic technical risks. 
%\newline
%\newline
%Created a realistic and actionable plan to increase satellite constellation average payload data throughput by 42\% yielding a 14\% increase in capacity (directly correlated with revenue).  Validated the plan with large scale year-in-the-life Python link budget modeling and systems engineering showing minimal schedule delay, and technical risk.
%\newline
%\newline
Joined just after Seed funding as the 12th employee.
\\
\\
\end{tabularx}

\begin{tabularx}{0.97\linewidth}{>{\raggedleft\scshape}p{2cm}X}
\gray Job Title & \textbf{Senior RF Systems Engineer}\\
\gray Employer & \textbf{BlackSky} \hfill Tukwilla, WA \& Remote\\
\gray Period & \textbf{April 2019 -- October 2021}\\
%total yearly compensation:$115k + 10k bonus + 0k bonus,, Senior RF Engineer, company_size=40
&
\vspace{-0.15 cm}
Created RF architecture diagrams, link budgets, test plans, and ran hands-on troubleshooting. Collaborated with customers and suppliers to design, manufacture, test, launch, and operate X (payload), S (TT\&C), GPS, and UHF-band space-based software defined radios linked to ground stations enabled by the AWS Ground Station service and the KSAT Lite ground station network.
%\newline
%\newline
%Managed cost, schedule, risk, regulatory compliance, and SWaP to stand up Low-Earth orbit small satellite constellations i%ncluding BlackSky, Loft Orbital, and NorthStar Earth \& Space.
\newline
\newline
Collaboratively designed, simulated, sourced, advised layout, and validated: parts, mixed signal PCB, connectors, cabling, and enclosure for a GPS RF system self-compatibility filter.  Multiple spacecraft successful in-orbit operation.
%\newline
%\newline
%Building an open source link budget model to simulate: throughput, coverage, power, and bandwidth.
%\newline
%\newline
%Awarded for saving \$0.5 million in recurring cost for flatsat constellation test benches with a deep dive into the technical specifications of the ground and space hardware, and concurrence with vendors.
\\
\\
\end{tabularx}

\begin{tabularx}{0.97\linewidth}{>{\raggedleft\scshape}p{2cm}X}
\gray Job Title & \textbf{RF Systems Engineer}\\
\gray Employer & \textbf{Kymeta} \hfill Redmond, WA\\
\gray Period & \textbf{February 2018 -- March 2019}\\
%total yearly compensation:$95k + 10k stock + 0k bonus, Systems Engineer, company_size=100
&
\vspace{-0.15 cm}
Developed and executed over-the-air combined OSI application, transport, network, and physical layer level test cases for a mobile Azure cloud connected MIMO Ku-band terminal with software defined phased array flat panel antennas and a DVB-S2 satellite modem
%\newline
%\newline
%Took on project management duties helping guide and educate team members towards a unified view of software processes, programming languages, and development tools, across Agile and Waterfall methodologies.
\newline
\newline
Wrote phased array antenna cross-polarization optimization algorithm in Python and integrated it with production level test codebase along with documentation, theoretical and actual response data.
\\
\\
\end{tabularx}

%\begin{tabularx}{0.97\linewidth}{>{\raggedleft\scshape}p{2cm}X}
%\gray Job Title & \textbf{RF Systems Software Engineer}\\
%\gray Employer & \textbf{Space Systems/Loral} \hfill Mountain View, CA\\
%\gray Period & \textbf{October 2016 -- January 2018}\\
%%total yearly compensation:$82k + 0k bonus + 0k stock, Senior Systems Engineer, company_size=4000
%&
%\vspace{-0.15 cm}
%Award wining role leading, developing, and managing a production Python client and services to exchange data between a PostgreSQL database storing 1 TB of antenna data and an RF downlink capacity tool.\
%\\
%\\
%\end{tabularx}

\begin{tabularx}{0.97\linewidth}{>{\raggedleft\scshape}p{2cm}X}
\gray Job Title & \textbf{Senior RF Systems Engineer}\\
\gray Employer & \textbf{Maxar} \hfill Mountain View, CA\\
\gray Period & \textbf{March 2015 -- January 2018}\\
%total yearly compensation:$82k + 0k bonus + 0k stock, Senior Systems Engineer, company_size=4000
&
\vspace{-0.15 cm}
%Led successful Forward downlink payload re-design, deployment, launch, in-orbit test, and handover of geostationary communication satellite Echostar 21 operating the forward payload receive at Ka-band and transmit at S-band.
%\newline
%\newline
%Award wining role leading, developing, and managing a production Python client and services to exchange data between a PostgreSQL database storing 1 TB of antenna data and an RF downlink capacity tool.
%\newline
%\newline
Wrote specifications, triaged vendors, reviewed test data collateral, and directed the installation, unit level and system level tests of the following passive and active RF units: diplexer, waveguide, directional coupler, band pass filter, low noise amplifier, downconverter, high power load, circulator, coaxial cable, master reference oscillator, and synthesizer.
\\
\\
\end{tabularx}

\begin{tabularx}{0.97\linewidth}{>{\raggedleft\scshape}p{2cm}X}
\gray Job Title & \textbf{Associate -> RF Systems Engineer}\\
\gray Employer & \textbf{Maxar} \hfill Mountain View, CA\\
\gray Period & \textbf{September 2013 -- March 2015}\\
%total yearly compensation:$77k + 0k bonus + 0k stock, Associate Systems Engineer, company_size=3000
%total yearly compensation:$68k + 0k bonus + 0k stock, Systems Engineer, company_size=3000
&
\vspace{-0.15 cm}
Developed Python analysis tool from scratch to model complex amplitude and time delay of 10,000+ RF units for ground-based beam-forming.
%\newline
%\newline
%Awarded by the CEO for saving \$0.25 Million and 3 weeks of production schedule with Python tool simulations.
\\
\\
\end{tabularx}

%\begin{tabularx}{0.97\linewidth}{>{\raggedleft\scshape}p{2cm}X}
%\gray Job Title & \textbf{Associate RF Systems Engineer}\\
%\gray Employer & \textbf{Space Systems/Loral} \hfill Mountain View, CA\\
%\gray Period & \textbf{June 2012 -- September 2013}\\
%total yearly compensation:$77k + 0k bonus + 0k stock, Associate Systems Engineer, company_size=3000
%total yearly compensation:$68k + 0k bonus + 0k stock, Systems Engineer, company_size=3000
%&
%\vspace{-0.15 cm}
% world’s highest capacity geostationary satellite
%Automated EIRP calculations for geostationary satellite Jupiter 2's return Ka-band downlinks at 32kW of transmit power.
%\newline
%Developed and maintained budgets analyzing RF channel performance over 80 unique countries during 1.5 year satellite design cycle.
%\\
%\\
%\end{tabularx}

%EDUCATION
\section{Education}
\begin{tabularx}{0.97\linewidth}{>{\raggedleft\scshape}p{2cm}X}
\gray Degree & \textbf{Bachelor of Science in Electrical Engineering}\\
%\gray Focus & \textbf{Applied Electromagnetics}\\
\gray University & \textbf{University of California, Davis} \\
\gray Period & \textbf{June 2009 -- June 2012}\\
\end{tabularx}


\end{center}
\end{document}
